\chapter{Discussion and Future Work}
\label{chap:conclusion}

This dissertation has contributed computational methods for fitting stochastic epidemic models to partially observed incidence and prevalence data. Despite the ubiquity and importance of this data setting, and the historical contributions of stochastic epidemic models to the study of disease transmission, there has remained a need for computational tools that are simple, broadly applicable, and robust. 

In Chapter \ref{chap:bda_for_fitting_sems_to_prevalence_data}, we developed an agent--based Bayesian data augmentation algorithm for fitting stochastic epidemic models to prevalence data in small to moderate size populations. Historically, agent--based data augmentation algorithms for fitting stochastic epidemic models have relied on reversible--jump Markov chain Monte Carlo schemes to sample subject--level disease histories. The data agnostic proposals used in these methods are inefficient and perform poorly in the absence of subject--level data, which has all but precluded their use in the analysis of epidemic count data. 