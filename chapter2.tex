\chapter{Background}
\label{chap:background}

\section{Compartmental Models for the Spread of Infectious Disease}
\label{sec:outbreak_models}

Compartmental epidemic models are used to describe the dynamics of an outbreak, to estimate how features of the population or environment affect its severity, and to understand how interventions might help to disrupt transmission. A compartmental model represents the time--evolution of an outbreak in terms of the disease histories of individuals as they transition between discrete states, or model compartments. When we use a compartmental model to describe the \textit{transmission} dynamics of an outbreak, the model compartments encode structural information about how individuals at different infection states interact to transmit the infectious agent. In contrast, states in a compartmental model for \textit{disease} dynamics typically correspond to discrete stages in the natural history of within--host disease progression without reference to the host's transmissive potential. This distinction is diagrammed in Figure \ref{fig:infec_vs_disease}.

Mechanistic compartmental models prescribe physical laws that govern the transmission dynamics of an outbreak. For example, the model might specify the rates of infectious contact between groups of people, or with an environmental reservoir that is a vector for transmission. Infection incidence and prevalence data are modeled conditionally on the mechanistic structure of the model, which specifies a functional form for the temporal and spatial evolution of the epidemic process. In contrast with their mechanistic counterparts, phenomenological models describe the data generating mechanism without explicit reference to the physical system that is under observation \cite{chowell2017fitting}. The price we pay for adopting a mechanistic approach is that our models will be obviously ``wrong", at least in the sense that all models are wrong, so it will be our responsibility to justifying their reasonableness. Our reward is that mechanistic models provide us with interpretable, multifaceted descriptions of outbreak dynamics. Moreover, the ubiquity of mechanistic models in the epidemiological literature enables us to incorporate information about specific aspects of transmission dynamics from other studies and settings into our own models. Historical overviews of mechanistic models for disease transmission may be found in \cite{anderson1992infectious,brauer2017mathematical,keeling2008,lessler2016mechanistic}.

Mechanistic compartmental models can be specified at varying levels of fidelity to the underlying epidemic process. Our models will range in granularity, from agent--based models in which we explicitly track the disease histories of individuals, to population--level models defined by the aggregate flow of individuals between model compartments. We refer to a mechanistic model as a stochastic epidemic model (SEM) when we model stochasticity in the epidemic process. Compartmental models with complex dynamics have historically represented the epidemic process as a deterministic system of ordinary differential equations \cite{keeling2008}. All of the models in this dissertation will treat the epidemic process as evolving continuously in time, but observed at discrete times. The decision to work with continuous--time models is advantageous when observation times are not evenly spaced, or when various sub--processes evolve on different time scales \cite{glass2003,shelton2014}. Discrete time models can be problematic when census interval of the data and generation time of the epidemic are misaligned, and generally produce results that are not independent of the choice time--step (see \cite{shelton2014} for examples). One (very) compelling reason to prefer discrete time models is that the computational effort required to fit a discrete time model is typically much lower than what is required in the continuous time setting. 

The task of fitting a SEM is complicated by the limited extent of epidemiological data, which are recorded at discrete observation times, commonly describe just one aspect of the disease process, e.g., infections, and usually capture only a fraction of cases. Complete subject--level data, which would consist of full transmission chains and exact times at which individuals transition through disease states, are often unavailable \cite{oneill2010}. Systematic underreporting, which may result from asymptomatic cases  makes it difficult to disentangle whether the data arose from a severe outbreak observed with low fidelity, or a mild outbreak where most cases were detected. This is all to say nothing of heterogeneities in outbreak dynamics or detection. Fitting SEMs in the absence of complete data can mildly be described as a complicated latent variable problem. 

Our objective will be to estimate the posterior distribution of parameters, $ \btheta $, that govern the dynamics of an outbreak, given the data $ \bY $. One challenge is that there are complicated dependencies in the data, hence
\begin{align*}
\pi(\btheta|\bY) &\propto\ L(\bY|\btheta)\pi(\btheta) \\
&= \prod_{\ell = 1}^{L} \pi(\bY_\ell|\bY_{1,\dots,\ell-1},\btheta)\pi(\btheta) \neq \prod_{\ell = 1}^{L} \pi(\bY_\ell|\btheta)\pi(\btheta).
\end{align*}
 Furthermore, the transition density, $ \pi(\bY_{\ell}|\bY_{1,\dots,\ell-1}) $, involves a high dimensional integral over the unobserved epidemic process, $ \bX $, 
\begin{equation}
\label{eqn:sem_post}
\pi(\btheta|\bY) \propto \int L(\bY|\bX,\btheta) \pi(\bX|\btheta)\pi(\btheta)\rmd\pi(\bX).
\end{equation}
This integral is analytically intractable when the state space of $ \bX $ is of even moderate size. 

In this dissertation, we use model the latent epidemic process using models that possess the Markov property, meaning that the forward--time evolution of the process depends on its history only through its current state. This choice simplifies the structure of (\ref{eqn:sem_post}) to that of a hidden Markov model where the data are conditionally independent given the latent epidemic process (diagrammed in Figure \ref{fig:semhmm}). The simplified posterior is  
\begin{align}
\label{eqn:sem_post_hmm}
\pi(\btheta|\bY) &\propto\int \prod_{\ell = 1}^{L} L\left (\bY_\ell|\bX(t_{\ell}),\btheta\right ) \pi\left (\bX(t_\ell)|\bX(t_{\ell-1}),\btheta\right )\pi(\btheta)\rmd\pi(\bX).
\end{align}
On its own, simplifying the structure of the SEM solves nothing. The integral in (\ref{eqn:sem_post_hmm}) is still analytically intractable. Moreover, the size of the state space of $ \bX $ presents additional challenges if we choose to work with the augmented posterior, 
\begin{equation}
\pi(\btheta,\bX|\bY)\propto L(\bY|\bX,\btheta)\pi(\bX|\btheta)\pi(\btheta).
\end{equation}

\begin{figure}[htbp]
	\centering
	\includegraphics[width=0.5\linewidth]{figures/SEM_HMM}
	\caption[Diagram of a Hidden Markov model.]{Diagram of a Hidden Markov model. The data are conditionally independent given the latent epidemic process.}
	\label{fig:semhmm}
\end{figure}


\subsection{An Agent--Based Susceptible--Infected--Recovered Model}
\label{subsec:sir_individual_mod}
For clarity of exposition, we will present the technical background on epidemic models in terms of the susceptible--infected--recovered (SIR model). Formal treatments that deal with this material in greater generality can be found in \cite{andersson2000stochastic,britton2018,brauer2008compartmental,fuchs2013inference,wilkinson2011stochastic}. The SIR model classifies individuals in a population of size $ N $ into one of three infection states: susceptible(S), infected (I), and recovered (R). Individuals are assumed to become infectious immediately upon entering the infected state, and acquire lasting immunity upon recovery. To simplify matters, we will assume the population is closed, meaning that there are no demographic changes or immigration, and that individuals are exchangeable. This latter assumption implies that individuals mix homogeneously and are alike in their infection dynamics. 

The SIR model defines an epidemic process, $ \bX = \lbrace\bX_1,\dots,\bX_N\rbrace $, that collects the subject--level subprocesses, $ \bX_j,\ j=1,\dots,N $, each of which takes values in the state space of disease state labels, $ \mcS_j= \lbrace S,I,R\rbrace $. A realized subject--path is of the form 
\begin{equation}
\bx_j = \left \lbrace\begin{array}{ll}
S,\ & \tau < \tau_I^{(j)},\\
I,\ & \tau_I^{(j)}\leq\tau<\tau_R^{(j)},\\
R,\ & \tau_R^{(j)} \leq \tau,
\end{array}\right .
\end{equation}
where $ \tau_I^{(j)} $ and $ \tau_{R}^{(j)} $ are the infection and recovery times for subject $ j $, and are possibly infinite (Figure \ref{fig:subjectsamplepaths}). The state space of $ \bX $ is  $ \mcS = \lbrace S,I,R\rbrace^N $, the Cartesian product of subject--level state labels. We denote by $ \bX(\tau) = \left (\bX_1(\tau),\dots,\bX_N(\tau)\right ) $ the state of $ \bX $ at time $ \tau $, and by $ \bX(\tau^+) $ the state just after time $ \tau $. 

\begin{figure}[htbp]
	\centering
	\includegraphics[width=0.8\linewidth]{figures/subject_sample_paths}
	\caption[Diagram of subject--level SIR paths.]{Diagram of subject--level paths (colored lines) for an SIR model in a population of size $ N=4 $. Individuals transition through infection states continuously in time. The epidemic process is defined in terms of the infection states of individuals in the population.}
	\label{fig:subjectsamplepaths}
\end{figure}

The waiting times between subject--level transition events are typically taken to be exponentially distributed. This will allow us to take advantage of several useful properties of exponential random variables (Figure \ref{fig:exp_props}, see \cite{wilkinson2011stochastic} for proofs). Critically, this choice also implies that $ \bX $ evolves as a continuous--time Markov chain (CTMC) with transition rate from configuration $ \bX $ to $ \bX^\prime $, differing only in the state of a single subject, given by
\begin{equation}
\lambda_{\bX,\bX^\prime} = \left \lbrace \begin{array}{rl}
\beta I,\ &\text{if } \bX\ \text{and } \bX^\prime\ \text{differ only in subject }j \text{, with }\bX_j=S\text{, and }\bX_j^\prime=I,\\
\mu,\ &\text{if } \bX\ \text{and } \bX^\prime\ \text{differ only in subject }j \text{, with }\bX_j=I\text{, and }\bX_j^\prime=R,\\
0,\ & \text{for all other configurations }\bX\ \text{and }\bX^\prime.
\end{array}\right.
\end{equation}
with per--contact infection rate,  $ \beta $, and recovery rate, $ \mu $. The quantity, $ 1/\mu $, is interpreted as the mean infectious period duration. That $ \bX $ is a \textit{Markov} process means that its forward--time evolution depends on its history only through its current state, i.e.,
\begin{equation}
\label{eqn:sir_markov}
\Pr\left (\bX(\tau + \dtau) = \bx^\prime | \lbrace\bX(\tau) = \bx,\ \tau\in[0,\tau]\rbrace, \btheta\right ) = \Pr\left (\bX(\tau + \dtau) = \bx^\prime | \bX(\tau) = \bx,\btheta\right ),
\end{equation}
where $ \bx^\prime,\bx \in \mcS$. $ \bX $ is \textit{time--homogeneous} since the rates of transition between configurations in the state space of $ \bX $ are constant over time. 

\begin{figure}[htbp]
	\caption{Very useful properties of exponential random variables.}
	\label{fig:exp_props}
	\begin{itemize}
		\item (Memoryless property) If $ Z\sim\mr{Exp}(\lambda)$, then $ \forall t,\dt\geq0 $  we have \begin{equation}\label{eqn:memoryless_prop}
		 \Pr(Z > t+\dt | Z>t) = \Pr(Z>\dt).
		\end{equation}
		\item (Racing exponentials) If $ Z_i\sim\mr{Exp}(\lambda_i),\ i=1,\dots,n $, are independent, then \begin{equation}\label{eqn:racing_exponentials}
		\underset{i}{\mr{min}}(Z_i) \sim \mr{Exp}\left (\lambda = \sum_i\lambda_i\right ).
		\end{equation}
		\item (Index of minimum) If $ Z_i\sim\mr{Exp}(\lambda_i),\ i=1,\dots,n $, are independent, then the index $ k $ of the minimum of $ Z_i $ is a random variable with probability mass function \begin{equation}\label{eqn:ind_of_min}
		\Pr(k|Z_k = \min(Z_1,\dots,Z_n)) = \frac{\lambda_k}{\sum_j\lambda_j}.
		\end{equation} 
	\end{itemize}
\end{figure}

Let $ \btau = \lbrace\tau_0,\dots,\tau_{K+1}\rbrace $, be the (ordered) set of $ K $ infection and recovery times of all individuals along with the endpoints of the time period $ [\tau_0,\ \tau_{K+1}] $. Let $ \ind{\tau_k \corresponds I} $ and $ \ind{\tau_k \corresponds R} $ indicate whether $ \tau_k $ is an infection or recovery time, and let $ \btheta = (\beta, \mu, \bX_0) $ denote the vector of parameters, including the initial state of $ \bX $ at time $ \tau_0 $. The CTMC likelihood of $ \bX $ over $ [\tau_0,\ \tau_{K+1}] $ is a product of exponential waiting time densities,
\begin{align} 
\label{eqn:sir_subj_likelihood}
L(\bX| \btheta) &= \prod_{k = 1}^{K}\left \lbrace \left [\beta I_{\tau_k}\times\ind{\tau_k \corresponds I} + \mu\times\ind{\tau_k \corresponds R}\right ] \exp{\left [-\left (\tau_k - \tau_{k-1}\right )\left (\beta I_{\tau_k} S_{\tau_k} + \mu I_{\tau_k}\right )\right ]}\right \rbrace \nonumber \\
& \hspace{0.2in} \times \exp \left [-\left (t_L - \tau_K\right )\left (\beta I_{\tau_K^+}S_{\tau_K^+} + \mu I_{\tau_K^+}\right )\right ]. 
\end{align}

In contrast with the population--level CTMC, $ \bX $, the subject--level subprocess, $ \bX_j $, is a \textit{time--inhomogeneous} CTMC with transition rate matrix
\begin{equation} 
\label{eqn:sir_subj_rate_mtx}
\bLambda^{j}(\btheta)(\tau) = \bordermatrix{ & S & I & R \cr
	S & -\beta I^{(-j)}(\tau) & \beta I^{(-j)}(\tau) & 0 \cr 
	I & 0 & -\mu & \mu \cr
	R & 0 & 0 & 0 },
\end{equation}
since $ I^{(-j)}(\tau) $, the number of infected individuals in the population at time $ \tau $, excluding individual $ j $, changes over time. We could also describe $ \bX_j $ as piecewise--homogeneous since the rate of infection for subject $ j $ is constant between times at which other individuals become infected and recover. 

\subsection{A Population--Level Susceptible--Infected--Recovered Model}
\label{subsec:sir_population_mod}
The subject--level SIR model is equivalent to an aggregated SIR model, expressed in terms of compartment counts, that is often preferred for various computational reasons \cite{allen2008introduction, andersson2000stochastic}. This equivalence derives from two properties of Markov processes, \textit{lumpability} and \textit{commutativity}. The population--level SIR model is usually presented for computational reasons since discarding the subject labels for infection and recovery events substantially reduces the computational burden of caching subject--level paths. We refer to \cite{tian2006lumpability} for a more formal presentation of the following discussion. 

Given a Markov process, $ \bX $ with state space $ \mcS = \lbrace s_1,\dots,s_P\rbrace $ and initial probability vector $ \pi $, we define another process, $ \overline{\bX} $ on the state space $ \overline{\mcS} = \lbrace S_1,\dots,S_\mathcal{L}\rbrace $, which is a \textit{partition} of $ \mcS $. In our setting, the partitioning will map a configuration in $ \mcS $ to a set in $ \overline{\mcS} $ by counting the number of people in each model compartment. The jump chain of the new process is obtained by partitioning the jump chain of the complete process. We want to establish conditions under which $ \overline{\bX} $ is stochastically coupled to $ \bX $. 

Suppose the initial distribution of $ \overline{\bX}(t_0) $, induced by the distribution of $ \bX(t_0) $, is \begin{equation*}
\Pr(\overline{\bX}(t_0) = S_i) = \mathrm{Pr}_\pi(\bX(t_0) \in S_i)
\end{equation*}
and that its transition probabilities are
\begin{equation*}
\Pr(\overline{\bX}(t+\Delta t) = S_j | \overline{\bX}(t)=\overline{\bx}(t^\prime), t^\prime \leq t) = \Pr(\overline{\bX}(t+\Delta t) \in S_j | \bX(t)=\bx(t^\prime), t^\prime \leq t),
\end{equation*}
where $ \overline{\bx}(t^\prime) $  and $ \bx(t^\prime) $ denote the paths of the complete Markov process and the new process. We say that the complete Markov process is \textit{lumpable} with respect to a partition, $ \overline{\mcS} $, of its state space if lumping the complete process results in a Markov process with respect to the lumped state space for every choice of $ \pi $ whose transition probabilities do not depend on $ \pi $. We refer to the process obtained by lumping as the \textit{lumped} Markov jump process (MJP). If the jump chain of the lumped MJP is the same as the lumped jump chain of the complete MJP, we say that the complete process is \textit{commutative} with respect to lumping.  

Let $ S_A $ and $ S_B $ be elements of $ \overline{\mcS} $, i.e., sets of states $ s_j \in \mcS $. The rate matrix of a CTMC is lumpable if
\begin{equation*}
\sum_{s_b \in S_B}\lambda_{s_a,s_b} = \sum_{s_b \in S_B}\lambda_{s_c,s_b}
\end{equation*}
for any pair of sets $S_A,S_B \in \overline{\mcS} $ and any pair of states  $(s_a, s_c) \in S_A$. A Markov process $ \bX $ is lumpable with respect to a partition of its state space if and only if its rate matrix is lumpable. Lumpability implies that transition probabilities for the lumped chain can be computed based on the lumped rate matrix  \cite{tian2006lumpability}.

Suppose $ \bX $ is lumpable with respect to the partition $ \overline{\mcS} $ of $ \mcS $. Then $ \bX $ is commutative with respect to the partition if and only if the rate matrix of $ \bX $ satisfies
$$\lambda_{s_a,s_b} = 0,\ \text{if}\ s_a,s_b\in S_A,\ \text{and} s_a\neq s_b. $$ In our context, commutativity means that the transition rate is zero between different configurations $ \bx_a $ and $ \bx_b $ with equal compartment counts. Commutativity implies that lumped quantities of interest for $ \bX $, such as transition rates or transition probabilities, can be equivalently computed based on $ \bX $, or based on the lumped process, $ \overline{\bX} $ \cite{tian2006lumpability}. 

\begin{figure}
	\centering
	\includegraphics[width=\linewidth]{figures/SIR_representations}
	\caption[Individual and lumped representations of SIR dynamics.]{Complete and lumped representations of SIR dynamics in a population of five individuals. The per--contact infectivity rate, $ \beta $, and the recovery rate, $ \mu $, parameterize exponential waiting time distributions between transition events. The complete Markov jump process evolves on the state space of subject state labels, $ \mcS = \lbrace S,I,R\rbrace^N $, with dynamics determined by the subject--level transition rates. Each susceptible may contact two infected individuals, while each infected individual recovers independently. The lumped process evolves on the state space of compartment counts, $ \widebar{\mcS} = \lbrace N_S,N_I,N_R:\ N_S + N_I+N_R=N\rbrace $, with dynamics determined by lumped transition rates. The waiting time distributions between transitions are derived by noting that if $ \tau_1\sim \exp(\lambda_1) $ and $ \tau_2\sim\exp(\lambda_2) $, then $ \tau_{\min} = \min(\tau_1,\tau_2)\sim\exp(\lambda_1+\lambda_2) $.}
	\label{fig:sirrepresentations}
\end{figure}

Turning back to the SIR model, we defined the epidemic process, $ \bX(\tau) = (\bX_1,\dots,\bX_N)$, with state space $ \mcS = \lbrace S, I, R\rbrace^N $. Let $ \bx = (x_1,\dots,x_N) $ denote a configuration of the state labels (e.g. $ \bx = (S, I, S, R, I) $), and let $$ \overline{\bx} = h(\bx) = \left (l=\sum_{i=1}^N\ind{x_i = S},m=\sum_{i=1}^N\ind{x_i=I},n = \sum_{i=1}^N\ind{x_i=R}\right ) $$ be the corresponding vector of compartment counts. The lumped state space is 
$$ \overline{\mcS} = \left \lbrace \overline{\bx} = (l,m,n): l,m,n \in \lbrace0,\dots,N\rbrace,\  l+m+n = N \right \rbrace, $$ 
which partitions $ \mcS $ by summing the number of individuals in each disease state. 

The population--level SIR model expressed in terms of compartment counts, $ \overline{\bX} = (S, I, R) \in \overline{\mcS} $ (depicted in Figure \ref{fig:sirrepresentations}), evolves as a CTMC on the lumped state space $ \overline{\mcS} $ with transition rates
\begin{equation*}
\begin{array}{cc}
\underline{\text{Transition}} & \underline{\text{Lumped Rate}} \\
(S,I,R) \longrightarrow (S-1,I+1,R) & \beta S I ,\\
(S,I,R) \longrightarrow (S,I-1,R+1) & \mu I .
\end{array}
\end{equation*}
To see how we arrive at these rates, note that in a population with $ S $ susceptibles, each of whom is independently infected at rate $ \beta I $, the time until the first infection is exponentially distributed with rate $ \lambda_{SI} = \beta SI $ by the racing exponentials property (\ref{eqn:racing_exponentials}). Similarly, the time to the first recovery in a population with $ I $ infected individuals is exponentially distributed with rate $\lambda_{IR} = \mu I $. Note that the mean infectious period duration is still $ 1/\mu $, as it was in the case of the subject--level SIR model.

Again, let $ \btau = \lbrace\tau_0,\dots,\tau_{K+1}\rbrace $, be the (ordered) set of $ K $ infection and recovery times, along with the endpoints of the time period $ [\tau_0,\ \tau_{K+1}] $ over which the outbreak is modeled. We indicate by $ \ind{\tau_k \corresponds I} $ and $ \ind{\tau_k \corresponds R} $ whether $ \tau_k $ is an infection or recovery time, and let $ \btheta = (\beta, \mu, \overline{\bX}_0) $ denote the vector of parameters, including the initial state of $ \overline{\bX} $ at time $ \tau_0 $. The CTMC likelihood of $ \overline{\bX} $ over $ [\tau_0,\ \tau_{K+1}] $ is a product of exponential waiting time densities,
\begin{align} 
\label{eqn:sir_pop_likelihood}
L(\overline{\bX}| \btheta) &= \prod_{k = 1}^{K}\left \lbrace \left [\beta S_{\tau_k} I_{\tau_k}\times\ind{\tau_k \corresponds I} + \mu I_{\tau_k}\times\ind{\tau_k \corresponds R}\right ] \exp{\left [-\left (\tau_k - \tau_{k-1}\right )\left (\beta I_{\tau_k} S_{\tau_k} + \mu I_{\tau_k}\right )\right ]}\right \rbrace \nonumber \\
& \hspace{0.2in} \times \exp \left [-\left (t_L - \tau_K\right )\left (\beta I_{\tau_K^+}S_{\tau_K^+} + \mu I_{\tau_K^+}\right )\right ]. 
\end{align} 

\subsection{A Dubiously Brief Review of CTMCs}
\label{subsec:ctmc_overview}

We briefly digress from our discussion of epidemic models to review some basic properties of CTMCs in a bit more generality. The following discussion is not intended to be comprehensive, but rather to provide an overview of some results that will be useful in this dissertation. We refer to \cite{bremaud1999markov,fuchs2013inference,guttorp1995stochastic,wilkinson2011stochastic} for more complete and rigorous discussions of the following material. For simplicity, we will focus on CTMCs with finite state spaces. 

The forward--time evolution of a CTMC is described by its transition kernel $$\Pr(\bX(\tau^\prime) = \bx^\prime | \bX(\tau) = \bx) = P_{\bx,\bx^\prime}(\tau,\tau^\prime),$$
where $ \bx,\bx^\prime\in\mcS $ and $ 0\leq\tau\leq\tau^\prime $. The transition kernel only depends on the time--elapsed when the process is time--homogeneous $ P_{\bx,\bx^\prime}(\tau,\tau^\prime) = P_{\bx,\bx^\prime}(|\tau^\prime - \tau|) \equiv P_{\bx,\bx^\prime}(\tau^\prime) $. In matrix form, the transition kernel, $ P $, is an $ r\times r $ stochastic matrix, where $ r = |\mcS| $, the rows of $ P $ sum to one, and $ P_{ii} = -\sum_{j\neq i}P_{ij}$. Trivially, $ P_{\bx,\bx^\prime}(0) = I $, i.e., there are no state changes in a time interval of zero length, almost surely. By the Markov property, the probability of transitioning from state $ i $ to state $ k $ in time $ s+t $ is 
\begin{equation}
\label{eqn:chapman_kolmogorov}
P_{ik}(s+t) = \sum_{j\in\mcS}P_{ij}(s)P_{jk}(t),
\end{equation}
or in matrix form, $$P(s+t) = P(s)P(t).$$
Equation (\ref{eqn:chapman_kolmogorov}) is known as the \textit{Chapman--Kolmogorov} master equation. 

The transition rate matrix (or \textit{infinitesimal generator}) of $ \bX $ is defined as the derivative of $ P(\tau) $ at $ \tau = 0 $, i.e.,
\begin{align*}
\label{eqn:ctmc_generator}
\bLambda &= \underset{\dtau \rightarrow 0}{\lim}\frac{P(\dtau) - P(0)}{\dtau} \\
&= \underset{\dtau \rightarrow 0}{\lim}\frac{P(\dtau) - I}{\dtau}, \\
\shortintertext{which implies that the infinitesimal transition matrix is} P(\dtau) &= I + \bLambda\dtau.
\end{align*}
Thus, for example, the infinitesimal transition probabilities for the population--level SIR model are
\begin{align*}
\Pr\left (\bX(\tau+\dtau) = (S-1, I+1, R) | \bX(\tau) = (S,I,R)\right ) &= \beta SI + o(\dtau), \\
\Pr\left (\bX(\tau+\dtau) = (S, I-1, R+1) | \bX(\tau) = (S,I,R)\right ) &= \mu I + o(\dtau).
\end{align*}
The transition probability matrix over an interval of arbitrary length, $ \tau $, solves the matrix differential equation,  \begin{equation}\label{eqn:kolmogorov_forward}
\deriv{}{\tau}P(\tau) = \bLambda P(\tau),\hspace{0.2in} s.t.\  P(0) = I.
\end{equation} 
Thus, $ P(\tau) = \exp(\bLambda\tau) $, can be computed using the matrix exponential. There are at least 19 dubious ways of computing the matrix exponential \cite{moler2003nineteen}. We will typically do so by diagonalizing $ \bLambda $ and exponentiating the eigenvalues. Section \ref{sec:mtx_exp} outlines this procedure in two separate cases: when $ \bLambda$ has real--valued eigenvalues, and also when the eigenvalues are complex. Equation (\ref{eqn:kolmogorov_forward}) is known as the \textit{forward} Kolmogorov equation, and is obtained by computing the derivative
\begin{align*}
\deriv{}{\tau}P(\tau) &= \frac{P(\tau + \dtau) - P(\tau)}{\dtau} \\
&= \frac{P(\dtau)P(\tau) - P(\tau)}{\dtau} \\
&= \frac{P(\dtau) - I}{\dtau}P(\tau)\\
&= \bLambda P(\tau).
\end{align*}
The \textit{backward} Kolmogorov equation is similarly obtained by
\begin{align}
\deriv{}{\tau}P(\tau) &= \frac{P(\tau + \dtau) - P(\tau)}{\dtau} \nonumber \\
&= \frac{P(\tau)P(\dtau) - P(\tau)}{\dtau} \nonumber \\
&= P(\tau)\frac{P(\dtau) - I}{\dtau}\nonumber\\
&= P(\tau)\bLambda .
\end{align}

\subsection{Inference and Computation for Stochastic Epidemic Models}
\label{subsec:sem_exact_inf}
Inference for SEMs based on CTMC representations of the epidemic process have historically fallen into three categories: Martingale methods, simulation--based methods, and data augmentation \cite{oneill2010}. 

Martingale methods estimate the parameters of interest using estimating equations based on martingales for counting processes embedded within the SEM, e.g., for infections and recoveries \citep{becker1977general, watson1981application, sudbury1985proportion, andersson2000stochastic, lindenstrand2013estimation}. However, these methods are not easily implemented for SEMs with complex dynamics fit to partially observed count data.

Simulation based methods use the SEM to generate latent epidemic paths that serve as the basis for inference. This class of methods includes approximate Bayesian computation (ABC) methods \citep{mckinley2009,toni2009}, pseudo--marginal methods \citep{mckinley2014simulation}, and sequential Monte Carlo (or particle filter) methods \cite{toni2009, andrieu2010particle, ionides2011iterated, dukic2012,koepke2016predictive,golightly2018efficient}. Within this class, the particle marginal Metropolis--Hastings (PMMH) algorithm of \cite{andrieu2010particle} stands out as a general method for Bayesian inference and is used as a benchmark method in Chapter \ref{chap:bda_for_fitting_sems_to_prevalence_data}. Although simulation--based methods have been used to fit complex models, the computational cost of simulating from CTMCs can become prohibitive for complex models. Furthermore, simulation--based methods suffer from well known pitfalls. ABC methods are sensitive to the choice of summary statistic, rejection threshold, and prior  \cite{toni2009}. Sequential Monte Carlo methods, on which pseudo--marginal methods often rely, are prone to ``particle impoverishment" problems \cite{cappe2006inference, dukic2012}. Examples of particle degeneracy are presented in supplementary Chapter \ref{chap:appendix_ch3}, and an inability to fit models  with complex dynamics with reasonable effort led us to abandon PMMH as a benchmark for the more complex models in Chapters \ref{chap:lna_for_sems} and \ref{chap:lna_extensions}. 

Agent--based data augmentation (DA) methods for fitting SEMs, first presented in \cite{gibson1998,oneill1999}, target the joint posterior distribution of the missing data and model parameters to obtain a tractable complete data likelihood. That the augmentation is agent--based refers to the introduction of subject--level disease histories, rather than population--level epidemic paths, as latent variables in the model. An advantage of the agent--based approach is that household structure and subject--level covariates may be incorporated into the model \cite{auranen2000,hohle2002,cauchemez2004bayesian, neal2004statistical,oneill2009}. Development of DA methods for SEMs is of continuing interest, and recent works by \cite{pooley2015}, \cite{QinShe15}, and \cite{shestopaloff2016sampling} have presented methods that could possibly be applied to epidemic count data. However, their algorithms forgo the flexibility of agent--based DA and, in the case of the latter two papers, have not been applied to SEMs.

\subsection{Large Population Approximations}
\label{subsec:sem_approximations}

It is often infeasible to work with the CTMC formulation of a SEM when modeling an outbreak in a large population, particularly when working within a Bayesian MCMC framework. The cardinality of model's state space grows polynomially in the population size. This makes it difficult to efficiently sample from the posterior or to explore the likelihood surface, even when fitting SEMs with relatively simple dynamics. A second challenge is that as the population size grows, so too do the numbers of transition events. Hence, repeatedly evaluating the likelihoods (\ref{eqn:sir_subj_likelihood}) and (\ref{eqn:sir_pop_likelihood}) is prohibitively expensive. 

Two commonly used approximations of the MJP representation of a SEM are through a system of ordinary differential equations (ODEs) and through a system of stochastic differential equations (SDEs). We will not go into detail on the derivations of these representations, except to give some intuition about the conditions under which they are appropriate. However, it is interesting to note that the Kolmogorov equations for the ODE, SDE, and MJP formulations of a SEM arise as special cases of a \textit{differential Chapman--Kolmogorov master equation} that generalizes (\ref{eqn:chapman_kolmogorov}) \cite{fuchs2013inference}. 

We presented the MJP representation of the SIR model as an \textit{extensive}, $ \bX $, where transition events led to jumps of size one in the compartment counts. We could have equivalently defined the model in terms of an \textit{intensive} process, $ \bXtil = \bX / N = (S/N,I/N,R/N) $, where transition events led to jumps of size $ 1/N $. As $ N\longrightarrow \infty$ , it becomes reasonable to consider approximating $ \bXtil $ by a process with continuous sample paths. The large population stochastic \textit{approximation} is the solution to an It\^{o} diffusion whose sample paths are continuous but nowhere differentiable. The infinite population deterministic functional \textit{limit} solves a system of ODEs and has smooth sample paths. We will present extensive forms of the SDE and ODE models, which are obtained by simply rescaling the intensive processes from which the approximations derive. 

\subsubsection{Deterministic representation as a system of ODEs}
\label{subsec:deterministic_models}

In the infinite population limit, sample paths of the SIR model are solutions to the following system of ODEs, subject to the initial constraint that $ \bX(\tau_0) = \bx_0 $:
\begin{align*}
\deriv{S}{t} &= -\beta S I \\
\deriv{I}{t} &= \beta S I - \mu I \\
\deriv{R}{t} &= \mu I.
\end{align*}
The state space of the ODE representation of the SIR model is $$ \mcS^R =  \lbrace(j,k,l):\ j,k,l\in[0,N], j+k+l = N\rbrace. $$ Note that the ODE model implies that if $ I(\tau) > 0 $ at any time $ \tau $, then $ I(\tau) > 0\ \forall\ \tau\in[0,\infty)$. Therefore, the ODE model is dubious if we are interested in answering questions about the stochastic emergence or extinction of an outbreak.

ODE formulations of SEMs are particularly useful, in part, because they more easily lend themselves mathematical analysis. One important quantity that is straightforwardly derived for ODE models is the basic reproduction number, $ R_0 $, which is interpreted as the expected number of secondary infections attributable to a single index case in an otherwise susceptible population (of infinite size). The basic reproduction number can be related to both the final size distribution and to the probability of a major outbreak \cite{allen2017primer,miller2012note}. Estimation of $ R_0 $ is more complicated for stochastic models, though it can be derived from a branching process approximation for the early behavior of an outbreak \cite{allen2008introduction}. In the case of deterministic models, $ R_0 $ is easily obtained as the spectral radius of the next generation matrix (NGM) for the linearized system of ODEs at the disease free equilibrium \cite{diekmann2009construction,van2017reproduction}. Other methods exist, e.g., by analyzing the survival function or by fitting phenomenological models when the growth rate of the outbreak is observed to be subexponential \cite{van2017reproduction}.

As an example, we will demonstrate the NGM method for computing $ R_0 $ for an outbreak with SIR dynamics where a fraction of the population, $ p_v $, is vaccinated and vaccine efficacy (VE)  modifies the per--contact rate of infection (VE for susceptibility: $ \nu_s $), the infectiousness of carriers (VE for infectiousnes{(s: $ \nu_i $), and the rate of recovery (VE for recovery: $ \nu_r $). We will write $ N^{u} = (1-p_v)N $ and $ N^{v} = p_vN $. The disease free equilibrium (DFE) is $$ \bX_{DFE} =  (S^{(u)}_{DFE} = \left ((1 - p_v)N,\ I^{(u)}_{DFE} = 0,\  R^{(u)}_{DFE} = 0,\ S^{(v)}_{DFE} = p_vN,\ I^{(v)}_{DFE} = 0,\ R^{(v)}_{DFE} = 0)\right ). $$ 
The linearized system of ODEs at the DFE is 
\begin{align*}
\deriv{I^{(u)}}{t} &= \beta \left (I^{(u)} + \nu_iI^{(v)}\right )N^{(u)} - \mu I^{(u)}\\
\deriv{I^{(u)}}{t} &= \beta \nu_s\left (I^{(u)} + \nu_iI^{(v)}\right )N^{(v)} - \mu\nu_r I^{(v)}
\end{align*}
The NGM is constructed as $K=-T\Sigma^{-1}$, where $T$ gives the rates of infectious contact between states at infection, $ I^{(u)} $ and $ I^{(v)} $, and $\Sigma$ contains the recovery rates out of states at infection. Here,
\begin{align*}
T &= \kbordermatrix{ & I^{(u)} & I^{(v)} \\
	I^{(u)} & \beta N^{u} & \nu_i\beta N^u\\
	I^{(v)} & \nu_s \beta N^{v} & \nu_s\nu_i\beta N^v}, & &\hspace{-2in}
\Sigma = \kbordermatrix{ & I^{(u)} & I^{(v)} \\
I^{(u)}&-\mu & 0 \\
I^{(v)}& 0 & -\mu\nu_r},\\
&\hspace{1.25in} K =\kbordermatrix{ & I^{(u)} & I^{(v)} \\
	I^{(u)} & \frac{\beta N^{u}}{\mu} & \frac{\nu_i\beta N^u}{\nu_r\mu}\\
	I^{(v)} & \frac{\nu_s \beta N^{v}}{\mu} & \frac{\nu_s\nu_i\beta N^v}{\nu_r\mu}}.
\end{align*}
Thus, $ R_0 = \sigma(K) = \frac{1}{2}\left (Tr(K) + \sqrt{Tr(K)^2 - 4\det(K)}\right )  = \frac{\beta N^{u}}{\mu} + \frac{\nu_s\nu_i\beta N^v}{\nu_r\mu}$.

\subsubsection{Diffusion approximations of Markov Jump Processes}
\label{subsubsec:diff_approx}

We will give the diffusion approximation for the SIR model after a, somewhat colloquial, review of diffusions based on material in \cite{fuchs2013inference,oksendal2003stochastic,wilkinson2011stochastic}. We begin with an SDE of the form \begin{equation}
\deriv{\bX_t}{t} = \bmu(\bX_t,t) + \bSigma(\bX_t,t)\bZ_t,\hspace{0.1in} t\geq s;\hspace{0.1in} \bX_x = \bx,\end{equation}
with \textit{drift vector} $ \bmu:\mathbb{R}^d\rightarrow\mathbb{R}^d $ and  \textit{diffusion matrix} $ \bSigma:\mathbb{R}^d\rightarrow \mathbb{R}^d\times\mathbb{R}^d $, which are interpretable as the infinitesimal first and second moments of the process innovations. $ \bZ_t $ is $ d $--dimensional Gaussian white noise. We denote by $ \rmd\bW_t = \bZ_t\dt$ standard $ d $--dimensional Brownian motion (see \cite{oksendal2003stochastic} for a formal definition). An \textit{It\^{o} diffusion} is a stochastic process, $ (\bX_t)_{t\geq0} $, that satisfies the It\^{o} stochastic integral equation
\begin{equation}
\bX_t = \bX_0 + \int_0^t\bmu(\bX_t,t)\dt + \int_0^t\bSigma(\bX_t,t)\rmd\bW_t,\end{equation}
which we can express equivalently in differential form,
\begin{equation}
\label{eqn:general_sde}
\rmd\bX_t = \bmu(\bX_t,t)\dt + \bSigma(\bX_t,t)\rmd\bW_t.
\end{equation}
The SDE (\ref{eqn:general_sde}) can be interpreted as the limit of a difference equation, $ \Delta\bX_t = \bmu(\bX_t,t)\Delta t + \bSigma(\bX_t,t)\Delta\bW_t $, with infinitely small time steps, $ \Delta t \rightarrow 0$. Hence, we can simulate approximate sample paths using an \textit{Euler--Maruyama} scheme by taking $ \Delta t = \epsilon>0 $ and sampling the innovations from ,
$$\Delta\bX_t\sim MVN\left (\bmu(\bX_t,t)\Delta t,\ \bSigma(\bX_t,t)\bSigma(\bX_t,t)^T\Delta t\right ).$$

Given a $ d $--dimensional It\^{o} process of the form (\ref{eqn:general_sde}), \textit{It\^{o}'s Lemma} gives us a formula for the SDE satisfied by a transformation of the original process. The general formula is as follows \cite{oksendal2003stochastic}: let $ g(\bx, t) = \left (g_1(\bx,t),\dots,g_p(\bx,t)\right ) $ be a twice continuously differentiable map from $ \mathbb{R}^d\times[0,\infty)\rightarrow\mathbb{R}^p $. Then the transformed process, $ \bY(\bW,t) = g(\bX_t, t)$, is also an It\^{o} process with component $ Y_k,\ k=1,\dots,p $, given by
$$\rmd Y_k, = \pdiv{g_k}{t}(\bX, t) + \sum_i\pdiv{g_k(\bX,t)}{x_i}\rmd X_i + \frac{1}{2}\sum_i\sum_j\pdiv{^2g_k(\bX,t)}{x_i\partial x_j}\rmd X_i\rmd X_j,$$
where $ \rmd W_i\rmd W_j = \dt\rmd W_i = \rmd W_i\dt = 0 $ and $ \rmd W_i\rmd W_i = \dt. $

The SDE approximation of a density dependent MJP can be rigorously derived in a number of ways. These include proving the convergence of the MJP Kolmogorov master equation to its SDE counterpart, convergence of the infinitesimal generator, and through a number of different system size expansions applied to the master equation \cite{fuchs2013inference}. An intuitive approach yielding an equivalent result, referred to as the \textit{Langevin approach}, postulates that the SDE approximation is obtained by matching the infinitesimal moments of the diffusion to those of the MJP \cite{fuchs2013inference,gillespie2000chemical,wallace2012linear}. 

We proceed by approximating the numbers of infections and recoveries in a small time interval, $ (t, t+\dt] $. Suppose that we can choose $ \dt $ such that the following two \textit{leap} conditions hold:
\begin{enumerate}
	\item $ \dt $ is sufficiently \textit{small} that the $ \bX^c $ is essentially unchanged over $ (t,t+\dt] $, so that the rates of infections and recoveries are approximately constant: 
	\begin{equation}\label{eqn:tau_cond_1_background}
	\blambda(\bX^c(t^\prime)) \approx \blambda(\bx^c(t)),\ \forall t^\prime \in (t,t+\dt].
	\end{equation}
	\item $ \dt $ is sufficiently \textit{large} that we can expect many disease state transitions of each type:
	\begin{equation}\label{eqn:tau_cond_2_backgroun}
	\blambda(\bx^c(t)) \gg \bs{1}.
	\end{equation}
\end{enumerate}
Condition (\ref{eqn:tau_cond_1_background}) can be trivially satisfied by choosing $ \dt $ to be infinitesimally small, and implies that the numbers of infections and recoveries in $ (t,t+\dt] $ are essentially independent of one another since the rates at which they occur are approximately constant within the interval \cite{gillespie2000chemical}. Furthermore, (\ref{eqn:tau_cond_1_background}) implies that the numbers of infections and recoveries in the interval are independent Poisson random variables with rates $ \blambda(\bx^c(t)\dt) $. For the SIR model, the numbers of infections and recoveries in an infinitesimal time increment are $ N_{SI}(\dt) \sim \mr{Poisson}(\beta S(t)I(t)\dt) $ and $ N_{IR}(t+\dt) \sim \mr{Poisson}(\mu I(t)\dt) $. Condition (\ref{eqn:tau_cond_2_background}), which we can reasonably expect to be satisfied in large populations where there are many infections and recoveries \cite{wallace2012linear}, implies that the Poisson distributed increments can be well approximated by independent Gaussian random variables. 

We now give the SDE approximation for the SIR model. Let $ \blambda(\bX) = (\lambda_{SI}, \lambda_{IR}) $ denote the rates at which individuals become infected and recover, and let $ \bA $ denote the matrix whose rows specify changes in counts of susceptible, infected, and recovered individuals corresponding to one infection or recovery event:
\begin{equation*}
\bA = \kbordermatrix{& S & I &  R\\
	S\rightarrow I& -1& 1 & 0\\
	I \rightarrow R & 0& -1 & 1
}.
\end{equation*}
The SDE for the SIR model is 
\begin{equation}
\label{eqn:sir_sde}
\rmd\bX(t) = \bA^T\blambda(\bX)\dt + \sqrt{\bA^T\diag(\blambda(\bX))\bA}\rmd\bW_t.
\end{equation}
This SDE, sometimes referred to as the \textit{chemical Langevin equation} (CLE), is given in a somewhat general form. If we wanted to change the model dynamics, the expressions for $ \bA $ and $ \blambda $ would differ, but the diffusion approximation would take the same form (with the caveat that $ \blambda $ must satisfy a set of Lipschitz conditions to ensure existence of the SDE, see \cite{fuchs2013inference,oksendal2003stochastic}). 

\subsubsection{Linear Noise Approximation}
\label{subsubsec:lna_background}

While it is straightforward to simulate sample paths for SDEs, the CLE transition density is generally intractable. We can approximate the CLE by Taylor expanding (\ref{eqn:sir_sde}) around its deterministic drift and discarding higher order terms. This leads to a decomposition of CLE paths as  Taylor expanding the SDE around its deterministic drift. 

\subsection{Computational Approaches to Fitting Stochastic Epidemic Models}
\label{sec:computational_background}

- computation with SDEs
- largely dependent on particle filter methods, cauchemez and ferguson use CIR process

\section{Bayesian Computation and Markov Chain Monte Carlo}
\label{sec:bayesian_computation}

\subsection{Markov Chain Monte Carlo}
\label{subsec:mcmc}

\subsubsection{Bayesian Data Augmentation}
\label{subsec:data_augmentation}

\subsubsection{Slice sampling}
\label{subsubsec:slice_sampling}

\subsubsection{Elliptical slice sampling}
\label{subsubsec:elliptical_slice_sampling}

\subsubsection{Adaptive multivariate normal slice sampling}
\label{subsubsec:mvn_slice_sampling}