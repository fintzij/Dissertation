
\Title{Bayesian Modeling of Partially Observed Epidemic Count Data}
\Author{Jonathan Fintzi}
\Year{2018}
\Program{Biostatistics}

\Chair{Vladimir Minin}{Co-chair}{}
\Chair{Jon Wakefield}{Co-chair}{}
\Signature{M. Elizabeth Halloran}
\Signature{James Hughes}

\copyrightpage

\titlepage  

\setcounter{page}{-1}
\abstract{
Epidemic count data reported by public health surveillance systems reflect the incidence or prevalence of an infectious agent as it spreads through a population, and are a primary source of information for informing response strategies and predicting how the outbreak is likely to spread. Incidence and prevalence counts are also often the only source of information about historical outbreaks, or outbreaks in resource limited settings, which are of interest for researchers seeking to develop an understanding of disease transmission during ``peace time", with an eye on preparing for future outbreaks. The absence of subject--level information, and the systematic underreporting of cases, makes it difficult to disentangle whether the data arose from a severe outbreak, observed with low fidelity, or a mild outbreak were most cases were detected. The magnitude of the missing data, and the dimensionality of the state space of the latent epidemic process, present challenges for fitting stochastic epidemic models. In this dissertation, we develop computational algorithms for fitting stochastic epidemic models to partially observed incidence and prevalence data in small and large population settings. Our algorithms are not specific to a particular set of model dynamics, but rather apply to a broad class of commonly used stochastic epidemic models, including models that allow for time--inhomogeneous transmission dynamics. We use our methods to analyze data from an outbreak of influenza in a British boarding school, to the 2014--2015 outbreak of Ebola in West Africa, and to the 2009--2011 A(H1N1) influenza pandemic in Finland.
}
 
\tableofcontents
\listoffigures
\listoftables 
 
\chapter*{Glossary}      % starred form omits the `chapter x'
\addcontentsline{toc}{chapter}{Glossary}
\thispagestyle{plain}
%
\begin{glossary}
\item[ACF] Autocorrelation function.
\item[BDA] Bayesian data augmentation
\item[CDF] Cumulative distribution function.
\item[CLE] Chemical Langevin equation.
\item[CLT] Central limit theorem.
\item[CP] Centered parameterization.
\item[CTMC] Continuous--time Markov chain.
\item[DA] Data augmentation.
\item[DFE] Disease free equilibrium.
\item[EliptSS] Elliptical slice sampler.
\item[ESS] Effective sample size.
\item[EVD] Ebola virus disease.
\item[FOI] Force of infection.
\item[GMRF] Gaussian Markov random field.
\item[ILI] Influenza--like illness.
\item[LNA] Linear noise approximation.
\item[MCSE] Monte Carlo standard error.
\item[MJP] Markov jump process.
\item[MMTL] Multinomial modification of the $ \tau $--leaping algorithm.
\item[MVNSS] Multivariate normal slice sampler.
\item[NGM] Next generation matrix.
\item[PACF] Partial autocorrelation function.
\item[PMCMC] Particle Markov chain Monte Carlo.
\item[PMMH] Particle marginal Metropolis--Hastings.
\item[PPI] Posterior predictive interval.
\item[PPP] Posterior predictive p--value.
\item[PSRF] Potential scale reduction factor.
\item[NCP] Non--centered parameterization.
\item[SDE] Stochastic differential equation.
\item[SEM] Stochastic epidemic model.
\item[SIR] Susceptible--infected--recovered model.
\item[TPM] Transition probability matrix.
\item[VE] Vaccine efficacy.
\item[WHO] World Health Organization.
\end{glossary}
 

\acknowledgments{ \vskip2pc
   {\narrower\noindent
   	To be written.
   \par}  
}

\dedication{\begin{center}
		To be written.
\end{center}}

