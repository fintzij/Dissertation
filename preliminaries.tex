
\Title{Bayesian Modeling of Partially Observed Epidemic Count Data}
\Author{Jonathan Fintzi}
\Year{2018}
\Program{Biostatistics}

\Chair{Vladimir Minin}{Co-chair}{}
\Chair{Jon Wakefield}{Co-chair}{}
\Signature{M. Elizabeth Halloran}
\Signature{James Hughes}

\copyrightpage

\titlepage  

\setcounter{page}{-1}
\abstract{
Epidemic count data reported by public health surveillance systems reflect the incidence or prevalence of an infectious agent as it spreads through a population. They are a primary source of information for shaping response strategies and for predicting how an outbreak will evolve. Incidence and prevalence counts are often the only source of information about historical outbreaks, or outbreaks in resource limited settings, which are of interest for researchers seeking to develop an understanding of disease transmission during ``peace time", with an eye on preparing for future outbreaks. The absence of subject--level information and the systematic underreporting of cases complicate the task of  disentangling whether the data arose from a severe outbreak, observed with low fidelity, or a mild outbreak were most cases were detected. The magnitude of the missing data and the high dimensional state space of the latent epidemic process present challenges for fitting epidemic models that appropriately quantify the stochastic aspects of the transmission dynamics. In this dissertation, we develop computational algorithms for fitting stochastic epidemic models to partially observed incidence and prevalence data. Our algorithms are not specific to particular model dynamics, but rather apply to a broad class of commonly used stochastic epidemic models, including models that allow for time--inhomogeneous transmission dynamics. We use our methods to analyze data from an outbreak of influenza in a British boarding school, the 2014--2015 outbreak of Ebola in West Africa, and the 2009--2011 A(H1N1) influenza pandemic in Finland.
}
 
\tableofcontents
\listoffigures
\listoftables 
 
\chapter*{Glossary}      % starred form omits the `chapter x'
\addcontentsline{toc}{chapter}{Glossary}
\thispagestyle{plain}
%
\begin{glossary}
\item[ACF] Autocorrelation function.
\item[BDA] Bayesian data augmentation
\item[CDF] Cumulative distribution function.
\item[CLE] Chemical Langevin equation.
\item[CLT] Central limit theorem.
\item[CP] Centered parameterization.
\item[CTMC] Continuous--time Markov chain.
\item[DA] Data augmentation.
\item[DFE] Disease free equilibrium.
\item[EliptSS] Elliptical slice sampler.
\item[ESS] Effective sample size.
\item[EVD] Ebola virus disease.
\item[FOI] Force of infection.
\item[GMRF] Gaussian Markov random field.
\item[ILI] Influenza--like illness.
\item[LNA] Linear noise approximation.
\item[MCSE] Monte Carlo standard error.
\item[MJP] Markov jump process.
\item[MMTL] Multinomial modification of the $ \tau $--leaping algorithm.
\item[MVNSS] Multivariate normal slice sampler.
\item[NGM] Next generation matrix.
\item[PACF] Partial autocorrelation function.
\item[PMCMC] Particle Markov chain Monte Carlo.
\item[PMMH] Particle marginal Metropolis--Hastings.
\item[PPI] Posterior predictive interval.
\item[PPP] Posterior predictive p--value.
\item[PSRF] Potential scale reduction factor.
\item[NCP] Non--centered parameterization.
\item[SDE] Stochastic differential equation.
\item[SEM] Stochastic epidemic model.
\item[SIR] Susceptible--infected--recovered model.
\item[TPM] Transition probability matrix.
\item[VE] Vaccine efficacy.
\item[WHO] World Health Organization.
\end{glossary}
 

\acknowledgments{ \vskip2pc
   {\narrower\noindent\onehalfspacing
   	The trouble with acknowledgements is that I don't have the words to convey how profoundly grateful I am to everyone who supported me throughout this journey. I hope this note of thanks will suffice.
   	
   	I am deeply indebted to my advisers, Vladimir and Jon, who taught me about the importance of resilience in research and life, and who were always strangely optimistic that we'd make it here, eventually. Vladimir and Jon are exceptional, not only for their breadth of knowledge and creativity, but also for their kindness and ability to see the humanity in those around them. They never made me feel less for my ignorance, even as I was often my own harshest critic. They always gave me new challenges at the moments I was prepared for them, even when I didn't believe I was. Vladimir and Jon, I could not have imagined better advisers. Truly, thank you for everything. 
   	
   	The work presented in this dissertation would not have been possible without the support from the Department of Biostatistics, and from the Center for Inference and Dynamics of Infectious Diseases (NIH/NIGMS MIDAS Center of Excellence U54-GM111274
   	), which funded me throughout the research years of my degree. In addition to supporting me financially during this period, CIDID provided me with incredible opportunities to meet researchers in the field of infectious disease modeling from all over the world, many of whom are listed in the bibliography of this dissertation. I especially want to thank Betz Halloran, not only for her contributions as a member of my dissertation committee and as the director of CIDID, but also for her support throughout my degree. Betz encouraged me to think more broadly about the scientific implications of my work and welcomed my participation in regular reading group meetings with the Center for Statistics and Quantitative Infectious Diseases at the University of Florida. These meetings were an invaluable opportunity to break out of my computational/statistical bubble and to learn about vaccine trials.
   	
   	I want to thank my other committee members, Jim Hughes and Neil Abernethy, for their feedback and support. Jim was also on my applied qualifying exam committee. One of the defining aspects of my graduate education was having the repeated experience of having Jim asking me sharp and insightful questions during each of my oral exams, which would invitably creep back into my thoughts for days after as I continued to try to come up with good answers. I also want to thank Scott Emerson, Galen Shorack, Ken Rice, Barbara McKnight, Lurges Inoue, Ali Shojaie, Thomas Fleming, Patrick Heagerty, Daniela Witten, Mathias Drton, Adam Spiro, Paul Sampson, Peter Hoff,  Katie Kerr, Emily Fox, and Brian Leroux who contributed to my education as either instructors or research mentors at various points.
   	
   	My ability to navigate graduate school was made manifestly easier by the dedicated staff who always had my back. I especially want to thank Stephanie Shadbolt and Rebecca Allen at CIDID, Gitana Garofalo and Sandra Coke, in the Department of Biostatistics, Ellen Reynolds in the Department of Statistics, and the computer support staff in both departments for all of their support and their hard work in helping me to make it through this in one piece. 
   	
   	I have spent more than my fair share of time in school, and had more than my fair share of teachers. With that said, I want to acknowledge two individuals who shaped my direction in life, my relationship with learning, and the person I have become. The first is Joan Liu, who was my sophomore year high school english teacher. Ms. Liu's greatest effect on me was not the result of any particular lesson, though she was an immensely talented teacher. Nor was it her infectious enthusiasm for literature, or her invaluable help navigating the college admissions process, for which I remain ever grateful. Ms. Liu transformed my relationship with learning when one morning she announced the unexpected passing of a classmate. I never realized that a student could matter so much to a teacher until I saw how deeply this shook her, and how deeply she cared. In her humanity, Ms. Liu changed my relationship with every teacher I would ever have, and by extension my ability to learn and to grow. I did not realize it at the time, but that moment changed my life.
   	
   	I would also like to acknowledge Irma Weiss, also a teacher, though never my teacher in a formal setting. Irma has become a dear friend and has taught me, mostly by her own example, so much about perseverence and compassion. I will always be grateful to her for everything she has done help me and my family grow wiser, more patient, and more kind. We are lucky to have such a teacher in our lives.
   	
   	I am fortunate to have had incredible friends in my life who, despite years of grad school induced neglect, insist that we are still friends. That they would persist in the face of such adversity is as much a testament to their insanity as mine. Thank you to my classmates at the UW. We suffered together a bit, but really, it was mostly a lot of fun! To all of my friends who I abandoned on the east coast, I'm coming home and I've got some new stats jokes! 
   	
   	Finally, I want to thank my family, without whom this process would have been infinitely more difficult. Thank you to my Uncle Yaki, who  patiently waited far too long to brag that I had finally finished school, and to my Aunt Joan for all of your wisdom throughout the years. To my family in Israel, there are too many of you to name, but please know that I love and miss you all, and that I am thinking of each of you. Donna, Anat, and Scott, you worked really hard to cheer me up and keep me on track during the darkest times of my dissertation. Donna, thank you for all the times you burned through the family data plan so I wouldn't procrastinate on the internet. Anat and Scott, you made Sybil just to cheer me up. That's a lot of work! But, in all seriousness, I couldn not have asked for better cheerleaders and I love you all. 
   	
   	Along the way to collecting my Ph.D. I got lucky and found a new family. Jamie, my love, it's hard to express how profoundly thankful I am to have you in my life. You helped me to find balance during times when I felt unsteady, and your good humor and optimism have provided me with no end of happy distractions. I can't wait for the rest of our life to start! Sam, Edna, and Danielle, thank you for so warmly welcoming me into your family, and for all of your support and encouragement throughout the last few years. 
   	
   	Most of all, thank you to my parents, Tilda and Ariel, better known as Imma and Abba. What successes I have had in my life would not have been possible without you. I know that it wasn't easy, and I know how much you sacrificed to make sure we had every opportunity to succeed. I'm so proud of you both. 
   \par}  
}

\dedication{\begin{center}
		To my parents, Ariel and Tilda, who dreamt their children could be anything they wanted. And to Jamie, my love, and my best friend.
\end{center}}

