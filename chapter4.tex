\chapter{Approximate Inference for Stochastic Epidemic Models of Outbreaks in Large Populations}
\label{chap:lna_for_sems}

\section{Overview}
\label{sec:lna_overview}

Surveillance and outbreak response systems often report incidence counts of new cases detected in each inter--observation time interval. Analyzing this type of time series data is challenging since we must overcome many of the same challenges that we face in modeling the transmission dynamics of infectious diseases in small population settings with prevalence data --- discrete snapshots of a continuously evolving epidemic process, detecting a fraction of the new cases, and often directly observing only one aspect of the disease process. Furthermore, our task is made more difficult by the additional computational burden that results from repeated evaluation of CTMC likelihoods; the products of exponential waiting time distributions consist of polynomially increasing numbers of terms, and agent--based data augmentation MCMC algorithms become unwieldy as the numbers of subject--path proposals required to meaningfully perturb the CTMC likelihood get large \citep{fintzi2017efficient}. 

In this chapter, we show how the LNA of Section \ref{subsubsec:lna_background} can be adapted to obtain approximate inference for SEMs fit to epidemic count data in large populations. Our contributions are threefold: First, we demonstrate how the SEM dynamics should be reparameterized so that the LNA can be used to approximate transition densities of the counting processes for disease state transition events. Second, we fold the LNA into a Bayesian data augmentation framework in which latent LNA paths are sampled using the elliptical slice sampling (EliptSS) algorithm of \cite{murray2010}. This provides us with general machinery for jointly updating the latent paths while absolving us of the \textit{de facto} modeling choice that the data be Gaussian in order to efficiently perform inference as in \cite{komorowski2009,fearnhead2014}, or the need to use particle filter methods for non--Gaussian emission distributions as in \cite{golightly2015delayed}. Finally, we introduce a non--centered parameterization for the latent LNA process that massively improves the efficiency of our DA MCMC framework and makes it tractable for fitting complex models. 

