\chapter{Appendix to Chapter 4}
\label{chap:appendix_ch4}

\section{Tuning the Initial Elliptical Slice Sampling Bracket Width}
\label{sec:lna_init_bracket_width}

\section{Choice of Estimation Scale and Implications for Mixing and Convergence}
\label{sec:est_scale_discussion}

\section{Specification of Initial Compartment Volumes}
\label{sec:lna_init_volumes}

\section{Simulation Details and Additional Results for Section \ref{subsec:lna_coverage}}
\label{sec:lna_coverage_supplement}

\subsection{Simulation Setup and MCMC Details}
\label{subsec:lna_coverage_setup_details}

In this simulation, repeated for each of the three different regimes of population size and initial conditions given in Table \ref{tab:lna_coverage_sim}, we simulated 500 datasets according to the following procedure:
\begin{enumerate}
	\item Draw $ \log(R0 - 1),\ 1/\mu,\ \logit(\rho),\ \log(\phi) $ from the priors given in Table \ref{tab:lna_coverage_sim}.
	\item Simulate an outbreak, $ \bN|\btheta $, under SIR dynamics from the MJP via Gillespie's direct algorithm \cite{gillespie1976general}. If there were fewer than 15 cases, simulate another outbreak. 
	\item Simulate the observed incidence, $ \bY|\bN,\btheta $, as a negative binomial sample of the true incidence in each epoch, i.e., $ Y_\ell\sim\mr{Neg.Binomial(\rho(N_{SI}(t_\ell) - N_{SI}(t_{\ell-1})), \phi)} $. If the outbreak died off before epoch 15, the dataset was truncated at 15 observations (i.e., the dataset consisted of a series of case counts accrued during the outbreak along with a series of trailing zeros accrued after the outbreak died off). If the outbreak lasted longer than 50 epochs, the dataset was truncated at 50 observations
\end{enumerate}

We proceed to fit SIR models using the LNA, ODE, and MMTL approximations. Priors for model parameters were assigned as in Table \ref{tab:lna_coverage_sim}. Five MCMC chains per model were initialized at random values near the true parameters and run for 35,000 iterations per chain. The first 10,000 iterations used to warm up each chain and adaptively estimate the empirical covariance matrix to be used in the multivariate Gaussian random walk Metropolis--Hastings proposals for parameters. The empirical covariance matrix was initialized as 0.01 times an identity matrix. After the warm--up period, the empirical covariance matrix was frozen and the final 25,000 iterations from each chain were combined to form the final MCMC sample. Convergence was assessed using potential scale reduction factors (PSRFs) \cite{brooks1998general}, computed via the \texttt{coda} R package \cite{codapackage}. PSRFs were less than 1.05 in cases.

For models fit via the LNA and ODE approximations, the covariance matrix was adapted as in algorithm 4 of \cite{andrieu2008tutorial}. The gain factor sequence was $\gamma_n = 0.25(1 + 0.05n)^{0.50001}$, and a small nugget variance, equal to 0.00001 times an identity matrix, was added during the adaptation phase. The target acceptance rate used in the adaptation was 0.234. The models were implemented using the \text{stemr} R package \cite{stemr}.

Inference via the MMTL approximation within PMMH were fit using the \texttt{pomp} R package \cite{pompjss}. We used 500 particles in the PMMH algorithm. This choice was made to mitigate issues of particle degeneracy that occurred with fewer particles for some datasets. The time step for MMTL was set to 1/7, which, for example, corresponds to $ \tau $--leaping over one day increments given weekly incidence data. The MCMC was initialized in the same way as LNA and ODE models, but the empirical covariance matrix was adapted according to a different cooling schedule. The gain factor sequence provided by the package is $ \gamma_n = n^\alpha $, where the cooling term, $ \alpha $, was set to 0.999. For some of the datasets, the PMMH algorithm degenerated during the adaptive phase of the MCMC. If this was the case, the MCMC was restarted at a different set of random initial conditions. The posterior sample consisted of the combined samples from all five MCMC chains after discarding the initial samples from the adaptation phase.

\subsection{Additional Results}
\label{subsec:lna_coverage_additional_results}

\begin{sidewaystable}[!ht]
	\small
	\centering
	\begin{tabular}{lcccccc}
		\hline
		Method & Parameter & Coverage & PMD & 95\% CIW & ESS & Rel. GM ESS/CPU time \\ 
		\hline
		LNA & log(R0) & 0.93 & 0 (-0.51, 0.55) & 1.04 (0.83, 1.36) & 1340 (274, 4029) & 0.63 (0.13, 3.4) \\ 
		LNA & log($\mu$) & 0.95 & -0.01 (-0.5, 0.48) & 0.93 (0.74, 1.13) & 1024 (199, 3988) & 0.48 (0.1, 2.91) \\ 
		LNA & logit($\rho$) & 0.93 & -0.05 (-1.44, 0.56) & 1.18 (0.54, 2.86) & 1225 (316, 3941) & 0.62 (0.17, 4.05) \\ 
		LNA & log($\phi$) & 0.96 & 0 (-0.66, 0.71) & 1.35 (0.92, 2.22) & 2021 (687, 4535) & 1 (0.31, 3.53) \\ 
		MMTL & log(R0) & 0.95 & 0.03 (-0.48, 0.55) & 1.09 (0.88, 1.39) & 7483 (5453, 9197) & --- \\ 
		MMTL & log($\mu$) & 0.95 & -0.03 (-0.51, 0.45) & 0.92 (0.75, 1.09) & 7481 (5442, 9197) & --- \\ 
		MMTL & logit($\rho$) & 0.94 & -0.03 (-1, 0.61) & 1.17 (0.51, 2.98) & 6725 (4328, 8506) & --- \\ 
		MMTL & log($\phi$) & 0.96 & -0.02 (-0.69, 0.71) & 1.33 (0.91, 2.21) & 7486 (5405, 9215) & --- \\ 
		ODE & log(R0) & 0.89 & -0.03 (-0.76, 0.56) & 1.03 (0.65, 1.38) & 6438 (4956, 7650) & 185 (105, 340) \\ 
		ODE & log($\mu$) & 0.86 & 0.05 (-0.51, 0.79) & 0.91 (0.53, 1.2) & 6420 (4908, 7663) & 184 (103, 344) \\ 
		ODE & logit($\rho$) & 0.72 & 0.12 (-1.03, 1.37) & 1.25 (0.48, 2.94) & 6237 (4346, 7556) & 203 (111, 401) \\ 
		ODE & log($\phi$) & 0.75 & -0.25 (-1.64, 0.54) & 1.25 (0.88, 2.11) & 6529 (5459, 7778) & 189 (116, 340) \\ 
		\hline
	\end{tabular}
	\caption{Detailed small population (N = 10,000) regime results for the coverage simulation presented in Section \ref{subsec:lna_coverage}. Models were fit via the linear noise approximation (LNA), multinomial modified $ \tau $--leaping (MMTL) within particle marginal Metropolis--Hastings, and deterministic ordinary differential equations (ODE). $ R_0 $ is the basic reproductive number of an outbreak, $ \mu $ is the recovery rate, $ \rho $ is the negative binomial case detection probability, $ \phi $ is the negative binomial over--dispersion parameter. We report the coverage rates of 95\% Bayesian credible intervals along with 50\% (2.5\%, 97.5\%) quantiles of posterior median deviations (PMD), 95\% credible interval widths (CIW), effective sample size (ESS), and relative geometric mean effective sample size per CPU time (Rel. GM ESS/CPU time).}
\end{sidewaystable}	

\begin{sidewaystable}[!ht]
	\small
	\centering
	\begin{tabular}{lcccccc}
		\hline
		Method & Parameter & Coverage & PMD & 95\% CIW & ESS & Rel. GM ESS/CPU time \\ 
		\hline
		LNA & log(R0) & 0.93 & -0.03 (-0.5, 0.49) & 0.93 (0.66, 1.28) & 2158 (397, 4981) & 0.67 (0.16, 3) \\ 
		LNA & log($\mu$) & 0.93 & 0.01 (-0.43, 0.41) & 0.83 (0.58, 1.07) & 1919 (307, 5276) & 0.63 (0.12, 3.05) \\ 
		LNA & logit($\rho$) & 0.95 & -0.01 (-1, 0.5) & 1.13 (0.47, 2.81) & 1704 (439, 4636) & 0.61 (0.18, 3.49) \\ 
		LNA & log($\phi$) & 0.94 & 0.01 (-0.57, 0.7) & 1.1 (0.79, 1.86) & 2916 (1179, 5442) & 1.02 (0.38, 3.1) \\ 
		MMTL & log(R0) & 0.95 & -0.01 (-0.46, 0.51) & 0.98 (0.67, 1.32) & 7284 (4713, 9107) & --- \\ 
		MMTL & log($\mu$) & 0.93 & -0.01 (-0.47, 0.37) & 0.82 (0.58, 1.05) & 7167 (4488, 8912) & --- \\ 
		MMTL & logit($\rho$) & 0.95 & -0.03 (-0.98, 0.56) & 1.09 (0.43, 2.96) & 6452 (3999, 8318) & --- \\ 
		MMTL & log($\phi$) & 0.94 & -0.02 (-0.59, 0.65) & 1.1 (0.79, 1.84) & 7096 (4640, 8999) & --- \\ 
		ODE & log(R0) & 0.86 & -0.02 (-0.56, 0.59) & 0.84 (0.52, 1.27) & 6592 (5213, 7771) & 187 (88, 354) \\ 
		ODE & log($\mu$) & 0.82 & -0.01 (-0.66, 0.47) & 0.73 (0.44, 1.07) & 6558 (5129, 7664) & 190 (87, 359) \\ 
		ODE & logit($\rho$) & 0.75 & -0.03 (-1.13, 0.85) & 0.88 (0.37, 2.75) & 6421 (5017, 7643) & 209 (107, 410) \\ 
		ODE & log($\phi$) & 0.82 & -0.17 (-1.36, 0.57) & 1.04 (0.78, 1.7) & 6637 (5452, 7755) & 193 (102, 365) \\ 
		\hline
	\end{tabular}
	\caption{Detailed medium population (N = 50,000) regime results for the coverage simulation presented in Section \ref{subsec:lna_coverage}. Models were fit via the linear noise approximation (LNA), multinomial modified $ \tau $--leaping (MMTL) within particle marginal Metropolis--Hastings, and deterministic ordinary differential equations (ODE). $ R_0 $ is the basic reproductive number of an outbreak, $ \mu $ is the recovery rate, $ \rho $ is the negative binomial case detection probability, $ \phi $ is the negative binomial over--dispersion parameter. We report the coverage rates of 95\% Bayesian credible intervals along with 50\% (2.5\%, 97.5\%) quantiles of posterior median deviations (PMD), 95\% credible interval widths (CIW), effective sample size (ESS), and relative geometric mean effective sample size per CPU time (Rel. GM ESS/CPU time).}
\end{sidewaystable}	

\begin{sidewaystable}[!ht]
	\small
	\centering
	\begin{tabular}{lcccccc}
		\hline
		Method & Parameter & Coverage & PMD & 95\% CIW & ESS & Rel. GM ESS/CPU time \\ 
		\hline
		LNA & log(R0) & 0.95 & -0.01 (-0.38, 0.51) & 0.77 (0.47, 1.24) & 3248 (638, 6127) & 1.13 (0.22, 3.99) \\ 
		LNA & log($\mu$) & 0.95 & 0.01 (-0.44, 0.33) & 0.67 (0.4, 1.02) & 3134 (499, 6038) & 1.12 (0.18, 3.95) \\ 
		LNA & logit($\rho$) & 0.93 & 0.01 (-0.8, 0.57) & 0.99 (0.35, 2.66) & 2514 (572, 5905) & 1.03 (0.24, 4.71) \\ 
		LNA & log($\phi$) & 0.95 & 0 (-0.44, 0.54) & 0.94 (0.67, 1.5) & 3987 (2201, 6184) & 1.63 (0.71, 3.95) \\ 
		MMTL & log(R0) & 0.93 & 0.03 (-0.38, 1.04) & 0.8 (0.31, 1.27) & 7030 (3789, 8915) & --- \\ 
		MMTL & log($\mu$) & 0.94 & -0.02 (-0.46, 0.32) & 0.65 (0.38, 0.98) & 6858 (3602, 8871) & --- \\ 
		MMTL & logit($\rho$) & 0.92 & 0.01 (-0.7, 0.65) & 0.95 (0.34, 2.64) & 6166 (3282, 7906) & --- \\ 
		MMTL & log($\phi$) & 0.95 & -0.03 (-0.51, 0.52) & 0.94 (0.66, 1.5) & 6266 (3735, 8960) & --- \\ 
		ODE & log(R0) & 0.89 & -0.01 (-0.37, 0.55) & 0.65 (0.36, 1.22) & 6828 (5566, 8025) & 193 (104, 417) \\ 
		ODE & log($\mu$) & 0.89 & 0 (-0.49, 0.34) & 0.55 (0.3, 1) & 6815 (5520, 7877) & 198 (105, 428) \\ 
		ODE & logit($\rho$) & 0.84 & -0.01 (-0.78, 0.68) & 0.74 (0.27, 2.62) & 6575 (5219, 7838) & 210 (118, 486) \\ 
		ODE & log($\phi$) & 0.91 & -0.08 (-0.59, 0.46) & 0.9 (0.64, 1.44) & 6721 (5571, 7683) & 209 (117, 413) \\ 
		\hline
	\end{tabular}
	\caption{Detailed large population (N = 250,000) regime results for the coverage simulation presented in Section \ref{subsec:lna_coverage}. Models were fit via the linear noise approximation (LNA), multinomial modified $ \tau $--leaping (MMTL) within particle marginal Metropolis--Hastings, and deterministic ordinary differential equations (ODE). $ R_0 $ is the basic reproductive number of an outbreak, $ \mu $ is the recovery rate, $ \rho $ is the negative binomial case detection probability, $ \phi $ is the negative binomial over--dispersion parameter. We report the coverage rates of 95\% Bayesian credible intervals along with 50\% (2.5\%, 97.5\%) quantiles of posterior median deviations (PMD), 95\% credible interval widths (CIW), effective sample size (ESS), and relative geometric mean effective sample size per CPU time (Rel. GM ESS/CPU time).}
\end{sidewaystable}	

\newpage
\begin{table}[!ht]
	\small
	\centering
	\begin{tabular}{lccc}
		Population size & ODE & LNA & MMTL \\ 
		\hline
		10,000 & 0.39 (0.21, 0.62) & 21.73 (10.83, 37.74) & 85.23 (42.31, 152.48) \\ 
		50,000& 0.42 (0.23, 0.62) & 32.27 (13.4, 55.8) & 88.36 (38.63, 153.54) \\ 
		250,000 & 0.45 (0.25, 0.78) & 33.08 (12.56, 70.86) & 87.4 (39.8, 166.87) \\
		\hline
	\end{tabular}
	\caption{Median (2.5\%, 97.5\%) quantiles of run times, in minutes, for MCMC chains in the coverage simulation presented in Section \ref{subsec:lna_coverage}. Models were fit via the linear noise approximation (LNA), multinomial modified $ \tau $--leaping (MMTL) within particle marginal Metropolis--Hastings, and deterministic ordinary differential equations (ODE).}
\end{table}	

\newpage

\section{Supplementary Simulations with Fixed Parameters}
\label{sec:lna_fixedpar_coverage}

\subsection{Simulation Setup}
\label{subsec:lna_fixedpar_setup}

The simulations presented in this section supplement the results of Section \ref{subsec:lna_coverage} in assessing the statistical and computation performance of the LNA approximation vis--a--vis the ODE and MMTL approximations. In contrast to the previous coverage simulation, here we will fix the model parameters to one of four regimes, presented in Table \ref{tab:lna_supplementary_coverage_sim}, that are characterized by either fast or moderate outbreak dynamics, and high or low detection probability. In each setting, we simulated 500 outbreaks from a MJP with SIR dynamics in a population of 50,000 individuals, five of whom were initially infected and the rest of whom were susceptible. The observed incidence in each epoch was a negative binomial sample of the true incidence. Outbreaks for which the number of observed cases was less than 25 were re--simulated. MCMC chains were tuned and SIR models were fit via the LNA, ODE, and MMTL approximations as described in Section \ref{subsec:lna_coverage_setup_details}. The initial comparment volumes were fixed at the true values. The models were fit using diffuse priors, also presented in Table \ref{tab:lna_supplementary_coverage_sim}. We caution that the priors used in this exercise are perhaps unreasonably diffuse, particularly in the context of epidemic modeling where prior information is often available, and that we expect credible intervals will be overly wide as a result (we would still expect nominal coverage to be incorrect, even under tighter priors, since the datasets were simulated under fixed parameter regimes).

\subsection{Results}
\label{subsec:lna_fixedpar_sim_results}

Coverage for credible intervals of ODE models tended to fall below nominal levels in spite of the bias towards wide intervals due to the diffusivity of the priors. This was particularly the case in parameter regimes 1 and 3, where the basic reproductive number was lower (and hence the simulated outbreak trajectories further from their thermodynamic limits). In these parameter regimes, coverage was particularly poor due as estimates of the outbreak dynamics tended to be farther from their true values and credible intervals were too tight and did not properly account for uncertainty about the parameter estimates, particularly those governing the measurement process. Coverage levels for models fit via the LNA and MMTL approximations exceeded their nominal levels as expected. 

ODE models remained the most computationally performant. However, in this exercise, the LNA substantially outperformed the MMTL approximation within PMMH in terms of ESS and ESS per CPU time. We believe this is largely attributable to the diffusivity of the priors, which not only fail to regularize the posterior, but likely pull it towards unreasonable regions of the parameter space. As a general comment, we would strongly caution practitioners against adopting such priors more broadly. While it may seem appealing to adopt such diffuse priors in pursuit of being "agnostic" to the underlying outbreak dynamics, one of the very good reasons for working within the Bayesian paradigm in this context is that we have quite a bit of prior information regarding the outbreak dynamics and reasonable ranges for the case detection probability. For example, we often have historic examples of outbreaks in similar settings that we can look to in specifying priors about the basic reproductive number. We also know, e.g., that humans do not live for centuries, so it makes little sense to assign a prior distribution for the mean infectious period duration that has mass on the scale of eons. 

\begin{table}[!ht]
	\label{tab:lna_supplementary_coverage_sim}
	\caption{Parameter regimes under which datasets were simulated and priors used to fit SIR models. Five hundred datasets were simulated for each of the parameter regimes  from a MJP with SIR dynamics. $ R0 = \beta N / \mu $ is the basic reproductive number and $ \mu $ is the recovery rate. The observed incidence was a negative binomial sample of the true incidence in each inter--observation interval with case detection probability $ \rho $ and overdispersion parameter $ \phi $.}\footnotesize
	\centering
	\begin{tabular}{lcccc}
		& \textbf{Regime 1} & \textbf{Regime 2} & \textbf{Regime 3} & \textbf{Regime 4} \\
		& \makecell{Low R0/Low $ \rho $} & \makecell{High R0/Low $ \rho $} & \makecell{Low R0/High $ \rho $} & \makecell{High R0/High $ \rho $} \\
		\hline
		\textbf{R0} & 1.75 & 3.25 & 1.75 & 3.25 \\ 
		$ \bs{\rho} $ & 0.25 & 0.25 & 0.75 & 0.75 \\
		$ \mu $ & 1 & 0.4 & 1 & 0.4 \\
		$ \phi $ & 5 & 5 & 5 & 5\\
		\hline
		&&&
	\end{tabular} 
	
	\begin{tabular}{cllc}
		\textbf{Parameter} & \textbf{Interpretation} & \textbf{Prior} & \textbf{Median (95\% Interval)} \\ \hline
		$ R0-1 $ & Basic reproduction \# - 1 & LogCauchy(0.4, 1) & $ \implies R0 = $ 2.50 (1.00, 4.9$ \times 10^5$) \\ 
		$ 1/\mu $ & Mean infectious period & LogCauchy(-0.7, 1)& 1.43 ($ 4.3\times 10^{-6},\ 4.7\times 10^5 $) \\
		$ \rho $ & Mean case detection prob. & Unif(0, 1) & 0.5 (0.025, 0.975) \\
		$ \phi $ & Neg.Binom. overdispersion & LogCauchy(1.5,1) & 4.48 (1.4$ \times 10^{-5},\ 1.5\times10^6 $)\\
		\hline
	\end{tabular}
\end{table}

\subsection{Results}
\label{subsec:lna_fixedpar_results}


\begin{sidewaystable}[ht]
	\small
	\centering
	\begin{tabular}{lcccccc}
		\hline
		Method & Parameter & Coverage & PMD & 95\% CIW & ESS & Rel. GM ESS/CPU time \\ 
		\hline
		LNA & log(R0) & 0.99 & 0.23 (-0.33, 0.85) & 1.9 (1.2, 3.49) & 1200 (251, 2508) & 19.1 (2.7, 76.4) \\ 
		LNA & $\log(\mu)$ & 0.98 & -0.21 (-0.81, 0.28) & 1.73 (1.04, 3.42) & 1086 (228, 2295) & 27.9 (3.7, 100.5) \\ 
		LNA & $\logit(\rho)$ & 0.98 & -0.06 (-0.46, 0.4) & 1.07 (0.73, 1.67) & 1233 (360, 2180) & 8.22 (2.03, 21.4) \\ 
		LNA & $\log(\phi)$ & 0.98 & 0.01 (-0.51, 0.74) & 1.32 (1.17, 1.71) & 2950 (1145, 4334) & 10.8 (2.1, 30.0) \\ 
		MMTL & log(R0) & 0.99 & 0.22 (-0.86, 0.87) & 4.4 (1.92, 53.26) & 232 (110, 445) & --- \\ 
		MMTL & $\log(\mu)$ & 1.00 & -0.22 (-0.79, 0.52) & 2.02 (1.26, 3.63) & 125 (58, 283) & --- \\ 
		MMTL & $\logit(\rho)$ & 0.99 & -0.06 (-0.47, 0.48) & 1.21 (0.83, 1.72) & 492 (258, 874) & --- \\ 
		MMTL & $\log(\phi)$ & 0.97 & -0.01 (-0.54, 0.71) & 1.32 (1.16, 1.72) & 942 (541, 1791) & --- \\ 
		ODE & log(R0) & 0.84 & 0.3 (-0.77, 2.43) & 1.78 (1.12, 5.93) & 3364 (252, 6513) & 3894 (169, 13949) \\ 
		ODE & $\log(\mu)$ & 0.78 & -0.24 (-2.57, 0.7) & 1.56 (0.93, 5.81) & 3307 (243, 6489) & 6454 (277.52, 15693) \\ 
		ODE & $\logit(\rho)$ & 0.78 & -0.08 (-0.67, 0.79) & 0.77 (0.51, 1.52) & 5225 (2398, 6933) & 2500 (869, 5719) \\ 
		ODE & $\log(\phi)$ & 0.94 & -0.1 (-0.73, 0.57) & 1.26 (1.14, 1.54) & 5702 (2768, 6991) & 1451 (532, 3117) \\ 
		\hline
	\end{tabular}
	\caption{Detailed results for the fixed parameter  simulation in which outbreaks and datasets were simulated under parameter regime 1, characterized by slow outbreak dynamics (R0 = 1.75) and low mean case detection probability ($ \rho=0.25 $). Models were fit via the linear noise approximation (LNA), multinomial modified $ \tau $--leaping (MMTL) within particle marginal Metropolis--Hastings, and deterministic ordinary differential equations (ODE). $ R_0 $ is the basic reproductive number of an outbreak, $ \mu $ is the recovery rate, $ \rho $ is the negative binomial case detection probability, $ \phi $ is the negative binomial over--dispersion parameter. We report the coverage rates of 95\% Bayesian credible intervals along with 50\% (2.5\%, 97.5\%) quantiles of posterior median deviations (PMD), 95\% credible interval widths (CIW), effective sample size (ESS), and relative geometric mean effective sample size per CPU time (Rel. GM ESS/CPU time).}
\end{sidewaystable}

\begin{sidewaystable}[ht]
	\small
	\centering
	\begin{tabular}{lcccccc}
		\hline
		Method & Parameter & Coverage & PMD & 95\% CIW & ESS & Rel. GM ESS/CPU time \\ 
		\hline
		LNA & log(R0) & 0.99 & -0.35 (-0.87, -0.05) & 2.01 (1.42, 3.03) & 1551 (399, 3065) & 64.2 (10.3, 265.1) \\ 
		LNA & $\log(\mu)$ & 0.99 & 0.34 (0.09, 0.8) & 1.9 (1.34, 2.92) & 1431 (368, 2865) & 23.8 (5.0, 101.7) \\ 
		LNA & $\logit(\rho)$ & 0.94 & 0.12 (-0.23, 0.55) & 1.02 (0.73, 1.71) & 1354 (509, 2357) & 13.1 (3.2, 40.6) \\ 
		LNA & $\log(\phi)$ & 0.96 & 0.02 (-0.53, 0.81) & 1.38 (1.25, 1.76) & 3286 (1756, 4442) & 19.5 (6.3, 50.5) \\ 
		MMTL & log(R0) & 0.99 & -0.41 (-1.11, -0.08) & 13.94 (2.95, 55.97) & 132 (31, 314) & --- \\ 
		MMTL & $\log(\mu)$ & 0.99 & 0.4 (0.12, 0.99) & 2.74 (2.13, 3.69) & 213 (90, 1944) & --- \\ 
		MMTL & $\logit(\rho)$ & 0.90 & 0.18 (-0.18, 0.64) & 1.78 (1.1, 2.73) & 375 (171, 737) & --- \\ 
		MMTL & $\log(\phi)$ & 0.97 & -0.03 (-0.59, 0.78) & 1.42 (1.25, 2) & 586 (315, 993) & --- \\ 
		ODE & log(R0) & 0.99 & -0.22 (-0.84, 0.65) & 1.85 (1.31, 3.58) & 3979 (1875, 5747) & 9412 (2368, 34873) \\ 
		ODE & $\log(\mu)$ & 0.98 & 0.21 (-0.78, 0.8) & 1.72 (1.21, 3.53) & 3976 (1851, 5752) & 3824 (9812, 12367) \\ 
		ODE & $\logit(\rho)$ & 0.85 & 0.05 (-0.34, 0.52) & 0.71 (0.5, 1.08) & 5024 (2846, 6390) & 2753 (1205, 6858) \\ 
		ODE & $\log(\phi)$ & 0.95 & -0.03 (-0.64, 0.67) & 1.33 (1.23, 1.6) & 5634 (4377, 6661) & 2027 (995, 4479) \\ 
		\hline
	\end{tabular}
	\caption{Detailed results for the fixed parameter  simulation in which outbreaks and datasets were simulated under parameter regime 2, characterized by fast outbreak dynamics (R0 = 3.25) and low mean case detection probability ($ \rho = 0.25 $). Models were fit via the linear noise approximation (LNA), multinomial modified $ \tau $--leaping (MMTL) within particle marginal Metropolis--Hastings, and deterministic ordinary differential equations (ODE). $ R_0 $ is the basic reproductive number of an outbreak, $ \mu $ is the recovery rate, $ \rho $ is the negative binomial case detection probability, $ \phi $ is the negative binomial over--dispersion parameter. We report the coverage rates of 95\% Bayesian credible intervals along with 50\% (2.5\%, 97.5\%) quantiles of posterior median deviations (PMD), 95\% credible interval widths (CIW), effective sample size (ESS), and relative geometric mean effective sample size per CPU time (Rel. GM ESS/CPU time).}
\end{sidewaystable}

\begin{sidewaystable}[ht]
	\small
	\centering
	\begin{tabular}{lcccccc}
		\hline
		Method & Parameter & Coverage & PMD & 95\% CIW & ESS & Rel. GM ESS/CPU time \\ 
		\hline
		LNA & log(R0) & 0.96 & 0.26 (-0.22, 0.93) & 1.68 (1.01, 3.21) & 1354 (198, 3054) & 16 (1.6, 70.2) \\ 
		LNA & $\log(\mu)$ & 0.96 & -0.23 (-0.85, 0.16) & 1.49 (0.84, 3.23) & 1210 (191, 2931) & 25.1 (2.8, 83.6) \\ 
		LNA & $\logit(\rho)$ & 0.99 & -0.2 (-0.96, 0.84) & 3.1 (1.61, 4.6) & 935 (393, 1893) & 6.9 (2.4, 18.7) \\ 
		LNA & $\log(\phi)$ & 0.97 & 0.02 (-0.52, 0.67) & 1.25 (1.12, 1.52) & 2851 (1425, 4283) & 13.4 (4.3, 31.6) \\ 
		MMTL & log(R0) & 0.99 & 0.24 (-0.48, 0.96) & 2.49 (1.44, 36.95) & 313 (139, 634) & --- \\ 
		MMTL & $\log(\mu)$ & 0.99 & -0.24 (-0.87, 0.24) & 1.75 (1, 3.44) & 141 (57, 359) & --- \\ 
		MMTL & $\logit(\rho)$ & 1.00 & -0.2 (-0.93, 0.86) & 3.48 (2.04, 4.99) & 439 (244, 711) & --- \\ 
		MMTL & $\log(\phi)$ & 0.97 & -0.02 (-0.53, 0.64) & 1.25 (1.12, 1.56) & 723 (372, 1376) & --- \\ 
		ODE & log(R0) & 0.79 & 0.34 (-0.41, 2.6) & 1.52 (0.95, 5.73) & 3688 (285, 6453) & 2951 (127, 10756.39) \\ 
		ODE & $\log(\mu)$ & 0.77 & -0.28 (-2.73, 0.36) & 1.31 (0.8, 5.69) & 3650 (263, 6457) & 5349 (177, 14237) \\ 
		ODE & $\logit(\rho)$ & 0.80 & -0.19 (-1.41, 1.57) & 2.27 (0.9, 4.66) & 3418 (1896, 5721) & 1767 (704, 4279) \\ 
		ODE & $\log(\phi)$ & 0.91 & -0.15 (-0.75, 0.48) & 1.16 (1.08, 1.34) & 5787 (2679, 7035) & 1754 (640, 4124) \\ 
		\hline
	\end{tabular}
	\caption{Detailed results for the fixed parameter simulation in which outbreaks and datasets were simulated under parameter regime 3, characterized by slow outbreak dynamics (R0 = 1.75) and high mean case detection probability ($ \rho = 0.75 $). Models were fit via the linear noise approximation (LNA), multinomial modified $ \tau $--leaping (MMTL) within particle marginal Metropolis--Hastings, and deterministic ordinary differential equations (ODE). $ R_0 $ is the basic reproductive number of an outbreak, $ \mu $ is the recovery rate, $ \rho $ is the negative binomial case detection probability, $ \phi $ is the negative binomial over--dispersion parameter. We report the coverage rates of 95\% Bayesian credible intervals along with 50\% (2.5\%, 97.5\%) quantiles of posterior median deviations (PMD), 95\% credible interval widths (CIW), effective sample size (ESS), and relative geometric mean effective sample size per CPU time (Rel. GM ESS/CPU time).}
\end{sidewaystable}

\begin{sidewaystable}[ht]
	\small
	\centering
	\begin{tabular}{lcccccc}
		\hline
		Method & Parameter & Coverage & PMD & 95\% CIW & ESS & Rel. GM ESS/CPU time \\ 
		\hline
		LNA & log(R0) & 1.00 & -0.27 (-0.68, 0.04) & 1.7 (1.24, 2.64) & 2217 (504, 4142) & 68.6 (10.0, 342.8) \\ 
		LNA & $\log(\mu)$ & 1.00 & 0.25 (-0.01, 0.62) & 1.62 (1.18, 2.63) & 2014 (493, 3894) & 26.4 (4.7, 85.4) \\ 
		LNA & $\logit(\rho)$ & 0.98 & 0.25 (-0.58, 1.35) & 3.49 (2.06, 4.63) & 1290 (563, 2572) & 14.8 (4.9, 59.9) \\ 
		LNA & $\log(\phi)$ & 0.96 & 0.03 (-0.52, 0.76) & 1.31 (1.2, 1.62) & 3506 (1910, 4689) & 27.0 (11.2, 66.1) \\ 
		MMTL & log(R0) & 1.00 & -0.35 (-0.93, -0.04) & 13.37 (1.83, 46.16) & 181 (48, 409) & --- \\ 
		MMTL & $\log(\mu)$ & 1.00 & 0.33 (0.06, 0.83) & 2.65 (1.61, 3.72) & 272 (128, 1994) & --- \\ 
		MMTL & $\logit(\rho)$ & 0.96 & 0.4 (-0.49, 1.57) & 4.49 (3.15, 7.21) & 309 (133, 545) & --- \\ 
		MMTL & $\log(\phi)$ & 0.97 & -0.06 (-0.63, 0.64) & 1.62 (1.26, 2.61) & 457 (213, 848) & --- \\ 
		ODE & log(R0) & 0.99 & -0.18 (-0.78, 0.84) & 1.67 (1.14, 3.73) & 4463 (1987, 6336) & 7879 (1915, 29567) \\ 
		ODE & $\log(\mu)$ & 0.99 & 0.18 (-0.99, 0.73) & 1.56 (1.07, 3.67) & 4424 (1952, 6312) & 3254 (867, 8316) \\ 
		ODE & $\logit(\rho)$ & 0.89 & 0.13 (-0.86, 1.71) & 2.59 (1.1, 4.57) & 3378 (1971, 5315) & 2271 (916, 7206) \\ 
		ODE & $\log(\phi)$ & 0.94 & -0.07 (-0.66, 0.62) & 1.26 (1.17, 1.48) & 5697 (4518, 6929) & 2490 (1211, 6558) \\ 
		\hline
	\end{tabular}
	\caption{Detailed results for the fixed parameter simulation in which outbreaks and datasets were simulated under parameter regime 4, characterized by fast outbreak dynamics (R0 = 3.25) and high mean case detection probability ($ \rho = 0.75 $). Models were fit via the linear noise approximation (LNA), multinomial modified $ \tau $--leaping (MMTL) within particle marginal Metropolis--Hastings, and deterministic ordinary differential equations (ODE). $ R_0 $ is the basic reproductive number of an outbreak, $ \mu $ is the recovery rate, $ \rho $ is the negative binomial case detection probability, $ \phi $ is the negative binomial over--dispersion parameter. We report the coverage rates of 95\% Bayesian credible intervals along with 50\% (2.5\%, 97.5\%) quantiles of posterior median deviations (PMD), 95\% credible interval widths (CIW), effective sample size (ESS), and relative geometric mean effective sample size per CPU time (Rel. GM ESS/CPU time).}
\end{sidewaystable}

\begin{table}[ht]
	\centering
	\begin{tabular}{lccc}
		\hline
		Parameter Regime & ODE & LNA & MMTL \\ 
		\hline
		Regime 1& 0.3 (0.21, 0.34) & 19.93 (14.96, 28.21) & 62.85 (54.16, 83.94) \\ 
		Regime 2& 0.28 (0.17, 0.36) & 14.41 (11.33, 20.43) & 48.89 (35.47, 67.2) \\ 
		Regime 3& 0.31 (0.2, 0.4) & 19.47 (15.57, 28.28) & 62.09 (55.75, 83.89) \\ 
		Regime 4& 0.28 (0.17, 0.36) & 14.66 (11.43, 20.58) & 49.04 (35.72, 67.53) \\ 
		\hline
	\end{tabular}
	\caption{Median (2.5\%, 97.5\%) quantiles of run times, in minutes, for MCMC chains in fixed parameter coverage simulations. Models were fit via the linear noise approximation (LNA), multinomial modified $ \tau $--leaping (MMTL) within particle marginal Metropolis--Hastings, and deterministic ordinary differential equations (ODE). True parameter values are presented in Table \ref{tab:lna_supplementary_coverage_sim}. Parameter regimes 1 and 3 had slower outbreak dynamics (R0 = 1.75, vs. R0 = 3.25). Parameter regimes 3 and 4 had higher case detection rates ($ \rho = 0.75 $, vs $ \rho = 0.25 $).}
\end{table}

\section{LNA Implementation Details and LNA Model Vignettes}
\label{sec:lna_implementation_vignettes}