\chapter{Introduction}
\label{chap:introduction}

\section{Motivating Examples}
\label{sec:motivating_examples}

\section{Compartmental Epidemic Models}
\label{sec:outbreak_models}

Compartmental epidemic models are used to describe the dynamics of an outbreak, to estimate how features of the population or environment affect its severity, and to understand how interventions might help to disrupt transmission. A compartmental model represents the time--evolution of an outbreak in terms of the disease histories of individuals as they transition between discrete states, or model compartments. When we use a compartmental model to describe the \textit{transmission} dynamics of an outbreak, the model compartments encode structural information about how individuals at different infection states interact to transmit the infectious agent. In contrast, states in a compartmental model for \textit{disease} dynamics typically correspond to discrete stages in the natural history of within--host disease progression without reference to the host's transmissive potential. This distinction is diagrammed in Figure \ref{fig:infec_vs_disease}.

A compartmental model is referred to as a mechanistic model when it prescribes physical laws that govern the transmission dynamics of an outbreak. For example, the model might specify the rates of infectious contact between groups of people, or with an environmental reservoir that is a vector for transmission. Infection incidence and prevalence data are modeled conditionally on the mechanistic structure of the model, which specifies a functional form for the temporal and spatial evolution of the epidemic process. In contrast with their mechanistic counterparts, phenomenological models describe the data generating mechanism without explicit reference to the physical system that is under observation \cite{chowell2017fitting}. The price we pay for adopting a mechanistic approach is that our models will be obviously ``wrong", at least in the sense that all models are wrong, so it will be our responsibility to justifying their reasonableness. Our reward is that mechanistic models provide us with interpretable, multifaceted descriptions of outbreak dynamics. Moreover, the ubiquity of mechanistic models in the epidemiological literature enables us to incorporate information about specific aspects of transmission dynamics from other studies and settings into our own models. Historical overviews of mechanistic models for disease transmission may be found in \cite{anderson1992infectious,brauer2017mathematical,keeling2008,lessler2016mechanistic}.

Mechanistic compartmental models can be specified at varying levels of fidelity to the underlying epidemic process. Our models will range in granularity, from agent--based models in which we explicitly track the disease histories of individuals, to population--level models defined by the aggregate flow of individuals between model compartments. We refer to a mechanistic model as a stochastic epidemic model (SEM) when we model stochasticity in the epidemic process. Compartmental models with complex dynamics have historically represented the epidemic process as deterministic systems of ordinary differential equations \cite{keeling2008}. All of the models in this dissertation will treat the epidemic process as evolving continuously in time, but observed at discrete times. The decision to work with continuous--time models is advantageous when observation times are not evenly spaced, or when various sub--processes evolve on different time scales \cite{glass2003,shelton2014}. Discrete time models can be problematic when the census interval of the data and the generation time of the epidemic are misaligned, and generally produce results that are not independent of the choice time--step (see \cite{shelton2014} for examples). One (very) compelling reason to prefer discrete time models is that the computational effort required to fit a discrete time model is typically much lower than what is required in the continuous time setting. 

The task of fitting a SEM is complicated by the limited extent of epidemiological data, which are recorded at discrete observation times, commonly describe just one aspect of the disease process, e.g., infections, and usually capture only a fraction of cases. Complete subject--level data, which would consist of full transmission chains and exact times at which individuals transition through disease states, are often unavailable \cite{oneill2010}. Systematic underreporting, which may result from asymptomatic cases or surveillance failures, makes it difficult to disentangle whether the data arose from a severe outbreak observed with low fidelity, or a mild outbreak where most cases were detected. This is all to say nothing of heterogeneities in outbreak dynamics or detection. Fitting SEMs in the absence of complete data can mildly be described as a complicated latent variable problem. 

\section{Organization of this dissertation}
\label{sec:organization}